
\documentclass[12pt]{amsart}
\usepackage{fullpage}
\usepackage{hyperref} % see geometry.pdf on how to lay out the page. There's lots.
% \geometry{landscape} % rotated page geometry

% See the ``Article customise'' template for come common customisations

\title{Syllabus: Phys 585 (Spring 2016)}
\date{} % delete this line to display the current date

%%% BEGIN DOCUMENT
\begin{document}

\maketitle

\section{Schedule}

Physics 585 meets once a week during Spring 2016: Tuesdays 12:00 noon - 1:00 pm, at CENPA (in Room 178, the conference room). It is offered as a graduate class for one credit (ungraded). Grad students, postdocs, and staff are welcome to attend and participate without registration, as this course is also CENPA's Journal Club for the quarter.

\section{Contact Information}
%\begin{table}[hl]
\begin{tabular}{lcl}
\textbf{Instructor}: Charlie Hagedorn &             & \textbf{Office}: CENPA/NPL 184 (tel. 206-543-1493) \\
\textbf{Email}: dparno@uw.edu &             &  \\
\hline
\end{tabular}\newline
%\end{table}%

\textbf{Public Course Website}: \url{http://faculty.washington.edu/dparno/Phys585.html}

\section{Course Objectives}

Eighty-six years ago, when Pauli proposed the existence of the neutrino, he famously lamented that it was a particle that could never be discovered. Now, neutrino measurements are routine, and moving to unprecedented levels of precision. Neutrino oscillations were our first hint of beyond-the-Standard-Model physics in the electroweak sector, and the neutrino sector may yet hold more surprises. In Spring 2016, we will survey open topics in neutrino physics. This quarter, students will:

\begin{itemize}
	\item	Read a variety of papers on active and historical areas of neutrino physics
	\item	Present at least one paper to their classmates
	\item	Actively discuss and critique historical and modern research progress
\end{itemize}

\section{When It's Your Turn To Present}
\begin{enumerate}
	\item	Select a paper or papers that you will present -- no more than 10 or 15 pages total. Suggested papers, along with review papers, are available at \url{http://faculty.washington.edu/dparno/neutrino_suggestedpapers.html}. Or, you can pick out your own favorite paper on your topic. Try to prioritize non-CENPA experiments. 
	\item	Email your chosen paper to the cenpa-journal mailing list by the preceding Friday.
	\item	Prepare and deliver a 15-30-minute presentation to explain the material to your classmates. You may find it useful to do some additional reading and research. You may choose between a chalk talk or a talk with presentation software.
	\item	After your presentation, lead the discussion, and answer questions as you are able.
\end{enumerate}

\section{When It Isn't Your Turn To Present}
\begin{enumerate}
	\item	Read the paper(s) selected by the presenter.
	\item	Write down a question about the paper for participation credit. Turn it in at the start of class, or email it to Charlie ahead of time. If the discussion stalls, random questions will be drawn to restart the conversation.
	\item	Listen attentively to your classmate's presentation.
	\item	Participate in the discussion. The discussion should be respectful even in cases of profound disagreement.
\end{enumerate}

\section{Course Policies}
\begin{itemize}
	\item	\textit{Travel:} If you must travel for your research, try to do the reading while you're away and send Charlie a discussion question so that you can still participate.
	\item	\textit{Switching dates:} If you cannot present on a date that you signed up for, please find a classmate willing to switch dates/topics with you.
	\item	\textit{Questions on a paper:} If you're having trouble figuring out some aspect of what you're presenting, Charlie is happy to help. Please contact her as far in advance of your talk as possible. Email is usually the most efficient communication tool.
	\item	\textit{Access and accommodations:} I want to give you the tools you need to succeed in this class. If you have already established accommodations with Disability Resources for Students (DRS), please communicate your approved accommodations to me at your earliest convenience so we can discuss your needs in this course. If you have not yet established services through DRS, but you have some temporary or permanent condition that requires accommodations, we recommend that you contact DRS (206-543-8924, uwdrs@uw.edu, disability.uw.edu). Either way that we will work together to coordinate reasonable accommodations.
	\item	\textit{Unforeseen circumstances:} Explain the situation and I will see what we can do.
\end{itemize}

\section{Schedule}
This schedule will be posted online at the course website, according to student signups. We will have eleven meetings over the course of the quarter.   The topics below may be covered in any order and of course we will not hit them all.

\begin{enumerate}
	\item Neutrino mass scale
	\item Historic neutrino measurements
	\item	Neutrino-nucleus interactions 
	\item	Neutrino oscillations: mass splittings and mixing angles
	\item Neutrino oscillations: mass hierarchy and CP violation
	\item	Neutrinoless double beta decay
	\item Solar neutrinos
	\item	Geoneutrinos
	\item	Supernova neutrinos
	\item	High-energy astrophysical neutrinos
	\item	Sterile neutrinos and neutrino anomalies
	\item	Exotic neutrino properties
\end{enumerate}


\section{Welcome to the Course!}

We're looking forward to exploring some of the wildest aspects of neutrinos together! From scattering experiments to mass measurements, from one end of the Standard Model to the other (and beyond), we hope you'll all enjoy the trip.

\end{document}